%%%%%%%%%%%%%%%%%%%%%%%%%%%%%%%%%%%%%%%%%%%%%%%%%%%%%%%
% Arquivo para entrada de dados para a parte pré textual
%%%%%%%%%%%%%%%%%%%%%%%%%%%%%%%%%%%%%%%%%%%%%%%%%%%%%%%
% 
% Basta digitar as informações indicidas, no formato 
% apresentado.
%
%%%%%%%
% Os dados solicitados são, na ordem:
%
% tipo do trabalho
% componentes do trabalho 
% título do trabalho
% nome do autor
% local 
% data (ano com 4 dígitos)
% orientador(a)
% coorientador(a)(as)(es)
% arquivo com dados bibliográficos
% instituição
% setor
% programa de pós gradução
% curso
% preambulo
% data defesa
% CDU
% errata
% assinaturas - termo de aprovação
% resumos & palavras chave
% agradecimentos
% dedicatoria
% epígrafe


% Informações de dados para CAPA e FOLHA DE ROSTO
%----------------------------------------------------------------------------- 
\tipotrabalho{Trabalho de Conclusão de Curso}
%    {Relatório Técnico}
%    {Dissertação}
%    {Tese}
%    {Monografia}

% Marcar Sim para as partes que irão compor o documento pdf
%----------------------------------------------------------------------------- 
 \providecommand{\terCapa}{Sim}
 \providecommand{\terFolhaRosto}{Sim}
 \providecommand{\terTermoAprovacao}{Sim}
 \providecommand{\terDedicatoria}{Nao}
 \providecommand{\terFichaCatalografica}{Sim}
 \providecommand{\terEpigrafe}{Nao}
 \providecommand{\terAgradecimentos}{Sim}
 \providecommand{\terErrata}{Nao}
 \providecommand{\terListaFiguras}{Sim}
 \providecommand{\terListaQuadros}{Nao}
 \providecommand{\terListaTabelas}{Sim}
 \providecommand{\terSiglasAbrev}{Nao} 
 \providecommand{\terSimbolos}{Nao}
 \providecommand{\terResumos}{Sim}
 \providecommand{\terSumario}{Sim}
 \providecommand{\terAnexo}{Nao}
 \providecommand{\terApendice}{Sim}
 \providecommand{\terIndiceR}{Nao}
%----------------------------------------------------------------------------- 

\titulo{Mais Que Uma Sala de Aula: O Impacto do Ensino Médio Integral Sobre Resultados Escolares Brasileiros}
\autor{Victor Manoel Alves}
\local{São Paulo - SP}
\data{2024} %Apenas ano 4 dígitos

% Orientador ou Orientadora
\orientador{}
%Prof Emílio Eiji Kavamura, MSc}
\orientadora{
Cristine Campos de Xavier Pinto}
% Pode haver apenas uma orientadora ou um orientador
% Se houver os dois prevalece o feminino.

% Em termos de coorientação, podem haver até quatro neste modelo
% Sendo 2 mulhere e 2 homens.
% Coorientador ou Coorientadora
\coorientador{}%Prof Morgan Freeman, DSc}
\coorientadora{}

% Segundo Coorientador ou Segunda Coorientadora
\scoorientador{}
%Prof Jack Nicholson, DEng}
\scoorientadora{}
%Prof\textordfeminine~Ingrid Bergman, DEng}
% ----------------------------------------------------------
\addbibresource{referencias.bib}

% ----------------------------------------------------------
\instituicao{%
Insper Instituto de Ensino e Pesquisa}

\def \ImprimirSetor{}%
%Setor de Tecnologia}

\def \ImprimirProgramaPos{}%Programa de Pós Graduação em Engenharia de Construção Civil}

\def \ImprimirCurso{}
%Curso de Engenharia Civil}

\preambulo{
Monografia apresentada ao programa de Graduação Em Economia como requisito para obtenção do rótulo de Bacharel em economia.}
%do grau de Bacharel em Expressão Gráfica no curso de Expressão Gráfica, Setor de Exatas da Universidade Federal do Paraná}

%----------------------------------------------------------------------------- 

\newcommand{\imprimirCurso}{Economia}
%Programa de P\'os Gradua\c{c}\~ao em Engenharia da Constru\c{c}\~ao Civil}

\newcommand{\imprimirDataDefesa}{
09 de Dezembro de 2018}

\newcommand{\imprimircdu}{
02:141:005.7}

% ----------------------------------------------------------
\newcommand{\imprimirerrata}{
Elemento opcional da \cites[4.2.1.2]{NBR14724:2011}. Exemplo:

\vspace{\onelineskip}

FERRIGNO, C. R. A. \textbf{Tratamento de neoplasias ósseas apendiculares com
reimplantação de enxerto ósseo autólogo autoclavado associado ao plasma
rico em plaquetas}: estudo crítico na cirurgia de preservação de membro em
cães. 2011. 128 f. Tese (Livre-Docência) - Faculdade de Medicina Veterinária e
Zootecnia, Universidade de São Paulo, São Paulo, 2011.

\begin{table}[htb]
\center
\footnotesize
\begin{tabular}{|p{1.4cm}|p{1cm}|p{3cm}|p{3cm}|}
  \hline
   \textbf{Folha} & \textbf{Linha}  & \textbf{Onde se lê}  & \textbf{Leia-se}  \\
    \hline
    1 & 10 & auto-conclavo & autoconclavo\\
   \hline
\end{tabular}
\end{table}}

% Comandos de dados - Data da apresentação
\providecommand{\imprimirdataapresentacaoRotulo}{}
\providecommand{\imprimirdataapresentacao}{}
\newcommand{\dataapresentacao}[2][\dataapresentacaoname]{\renewcommand{\dataapresentacao}{#2}}

% Comandos de dados - Nome do Curso
\providecommand{\imprimirnomedocursoRotulo}{}
\providecommand{\imprimirnomedocurso}{}
\newcommand{\nomedocurso}[2][\nomedocursoname]
  {\renewcommand{\imprimirnomedocursoRotulo}{#1}
\renewcommand{\imprimirnomedocurso}{#2}}


% ----------------------------------------------------------
\newcommand{\AssinaAprovacao}{

\assinatura{%\textbf
   {Naercio Aquino Menezes Filho \\ Insper}}
      
   \begin{center}
    \vspace*{0.5cm}
    %{\large\imprimirlocal}
    %\par
    %{\large\imprimirdata}
    \imprimirlocal\\
    2024
    \vspace*{1cm}
  \end{center}
  }
  
% ----------------------------------------------------------
%\newcommand{\Errata}{%\color{blue}
%Elemento opcional da \textcite[4.2.1.2]{NBR14724:2011}. Exemplo:
%}

% ----------------------------------------------------------
\newcommand{\EpigrafeTexto}{%\color{blue}
\textit{``Não vos amoldeis às estruturas deste mundo, \\
		mas transformai-vos pela renovação da mente, \\
		a fim de distinguir qual é a vontade de Deus: \\
		o que é bom, o que Lhe é agradável, o que é perfeito.\\
		(Bíblia Sagrada, Romanos 12, 2)}
}

% ----------------------------------------------------------
\newcommand{\ResumoTexto}{%\color{blue}
    Este trabalho de conclusão de curso explora os efeitos da ampliação da jornada escolar sobre o desempenho acadêmico dos alunos. A pesquisa investiga como o aumento do tempo de permanência dos alunos na escola afeta seu desempenho em testes padronizados e em indicadores educacionais selecionados. A análise é realizada com base em uma revisão abrangente da literatura global e evidências específicas para o contexto brasileiro. Para obtenção do efeito causal do programa Ensino Médio Integral sobre alunos brasileiros, adota-se a metodologia de \textit{staggered difference-in-differences} sobre dados públicos do Instituto Nacional de Estudos e Pesquisas Educacionais Anísio Teixeira, ainda não aplicada para avaliar o programa a nível nacional. Os resultados sugerem melhoras ao longo do tempo de exposição sobre as notas dos alunos nas provas do Exame Nacional do Ensino Médio de escolas participantes. Os resultados também apontam para melhoras nas taxas de abandono e aprovação para escolas participantes.} 

\newcommand{\PalavraschaveTexto}{%\color{blue}
Programa Ensino Médio Integral, Desempenho Acadêmico, \textit{Staggered difference-In-differences}.}

% ----------------------------------------------------------
\newcommand{\AbstractTexto}{%\color{blue}
This undergraduate thesis explores the effects of extending the school day on students' academic performance. The research investigates how increasing students' time spent in school affects their performance on standardized tests and selected educational indicators. The analysis is based on a comprehensive review of global literature and evidence specific to the Brazilian context. To obtain the causal effect of the Full High School program on Brazilian students, the methodology of staggered difference-in-differences is adopted using publicly available data from the Instituto Nacional de Estudos e Pesquisas Educacionais Anísio Teixeira, which has not yet been applied to evaluate the program at the national level. The results suggest improvements over time in students' scores on the Exame Nacional do Ensino Médio for participating schools. There are also improvements in dropout rates and approval rates for participating schools.
}
% ---
\newcommand{\KeywordsTexto}{%\color{blue}
Full-Time High School Program, Academic Performance, Staggered difference-in-differences.
}

% ----------------------------------------------------------
\newcommand{\Resume}
{%\color{blue}
%Il s'agit d'un résumé en français.
} 
% ---
\newcommand{\Motscles}
{%\color{blue}
 %latex. abntex. publication de textes.
}

% ----------------------------------------------------------
\newcommand{\Resumen}
{%\color{blue}
%Este es el resumen en español.
}
% ---
\newcommand{\Palabrasclave}
{%\color{blue}
%latex. abntex. publicación de textos.
}

% ----------------------------------------------------------
\newcommand{\AgradecimentosTexto}{%\color{blue}

Expresso minha sincera gratidão à minha avó, Maria, e à minha mãe, Cristiane, pelo sacrifício e apoio incansável que tornaram possível minha jornada até este ponto. À minha querida irmã, Vitória, agradeço pela presença constante e pelo apoio inestimável ao longo desta trajetória.

A Deus e aos orixás pela proteção e resiliência que me permitiram trilhar esta jornada.

Aos membros do grupo Green Bay ANPECkers, meu sincero obrigado por nossa jornada conjunta. Aos amigos de escola e ao meu irmão de vida, Paulo, agradeço pela colaboração e pelo sucesso que alcançamos juntos. Um agradecimento especial à Ana Flavia pelo seu constante companheirismo.

A todos os mentores que encontrei ao longo de minha trajetória acadêmica e profissional. Cristine, Thiago, Ricardo, Adriano, Davi e Thais, é um privilégio ter vocês como inspiração de profissional que desejo ser. E ao Programa de Bolsas, agradeço por acreditar em meu potencial e por proporcionar a oportunidade incrível e altamente impactante que vivenciei aqui.
}

% ----------------------------------------------------------
\newcommand{\DedicatoriaTexto}{%\color{blue}
\textit{ Este trabalho é dedicado às crianças adultas que,\\
   quando pequenas, sonharam em se tornar cientistas.}
	}

